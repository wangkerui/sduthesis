\chapter{逍遥游}
\echapter{Xiao Yao You}

北冥(míng)有鱼,其名为鲲(kūn)。鲲之大,不知其几千里也。化而为鸟,其名为鹏。鹏之背,不知其几千里也;怒而飞,其翼若垂天之云。是鸟也,海运则将徙(xǐ)于南冥。南冥者,天
池也。齐谐者,志怪者也。谐之言曰:“鹏之徙于南冥也,水击三千里,抟(tuán)扶摇而上者九万里,去以六月息者也。”野马也,尘埃也,生物之以息相吹也。天之苍苍,其正色邪(yé)?其远
而无所至极邪?其视下也,亦若是则已矣。且夫水之积也不厚,则其负大舟也无力。覆杯水于坳(ào)堂之上,则芥(jiè)为之舟,置杯焉则胶,水浅而舟大也。风之积也不厚,则其负大翼也
无力。故九万里,则风斯在下矣,而后乃今培风;背负青天,而莫之夭阏(yāo è)者,而后乃今将图南。蜩(tiáo)与学鸠笑之曰:“我决(xuè)起而飞,抢(qiāng)榆枋(fāng),时则不至,
而控于地而已矣,奚以之九万里而南为?”适莽(mǎng)苍者,三餐而反,腹犹果然;适百里者,宿(xiu)舂(chōng)粮;适千里者,三月聚粮。之二虫又何知!

小知(zhì)不及大知(32),小年不及大年。奚以知其然也?朝(zhāo)菌(jūn)不知晦朔(huì shuò),蟪蛄不知春秋,此小年也。楚之南有冥灵者,以五百岁为春,五百岁为秋;上古有大椿
(chūn)者,以八千岁为春,八千岁为秋,此大年也。而彭祖乃今以久特闻,众人匹之,不亦悲乎!汤之问棘也是已。(汤问棘曰:‘上下四方有极乎?’棘曰:'无极之外复无极也。【注】)穷发
(fà)之北,有冥海者,天池也。有鱼焉,其广数千里,未有知其修者,其名为鲲。有鸟焉,其名为鹏,背若泰山,翼若垂天之云,抟扶摇羊角而上者九万里,绝云气,负青天,然后图南,且适
南冥也。斥鴳(yàn)笑之曰:‘彼且奚适也?我腾跃而上,不过数仞而下,翱翔蓬蒿之间,此亦飞之至也。而彼且奚适也?’”此小大之辩也。

故夫知(zhì)效一官,行比一乡,德合一君,而(néng)征一国者,其自视也亦若此矣。而宋荣子犹然笑之。且举世而誉之而不加劝,举世而非之而不加沮(jǔ),定乎内外之分,辩乎荣辱之境,
斯已矣。彼其于世,未数(shuò)数(shuò)然也。虽然,犹有未树也。夫列子御风而行,泠(líng)然善也,旬有(yòu)五日而后反。彼于致福者,未数数然也。此虽免乎行,犹有所待者也。
若夫乘天地之正,而御六气之辩,以游无穷者,彼且恶乎待哉?故曰:至人无己,神人无功,圣人无名。

