%%% Local Variables:
%%% mode: latex
%%% TeX-master: ../main.tex
%%% End:

% \e<sectionlevel> (如 \echapter, \esection, \esubsection) 用于生成英文目录
\chapter{一级标题}
\echapter{English Chapter Name}
\label{cha:cha}
示例章节命令和参考文献的用法。注意:本文档仅是示例,请一定先阅读详细的说明文档 sduthesis.pdf 。


\section{二级标题}
\esection{English Section Name}
这是二级标题。

所有级别的标题都可以使用带 e 的版本生成英文目录(四级标题除外)。

\subsection{三级标题}
\esubsection{English Subsection Name}
这是三级标题。

\subsubsection{四级标题}
\esubsubsection{Si Ji Biao Ti}
四级标题不会出现在目录中。

另外,上述标题带 * 号的版本都不带编号,不计入目录。

\subsubsection*{starsubsubsection}
本节是带 * 号版本的例子。


\section{引用用法}
\esection{References}
参考文献有两种用法:一是上标模式,如这是上标模式\cite{cnproceed},二是正文模式,如文献\onlinecite{cnarticle}指出...。

\section{本文的结构}
\esection{Paper Structure}
本文第\ref{cha:cha}章示例章节命令和参考文献的用法。

本文第\ref{cha:example}章示例正文的用法,如插图和表格,公式和算法,定义、定理和证明环境等。

%%%
%%% End of File
%%%
